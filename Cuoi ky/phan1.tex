\textcolor{blue}{\section{Introduction}}
Lý thuyết số đóng một vai trò quan trọng trong mật mã học,
chủ thể của việc biến đổi thông tin để thông tin trở nên không dễ dàng khôi phục được
nếu không có hiểu biết đặc biệt.
Lý thuyết số là cơ sở của nhiều cách mã hóa cổ điển,
được sử dụng lần đầu tiên cách đây hàng nghìn năm,
và được sử dụng rộng rãi cho đến thế kỷ 20.
Các cách mã hóa này mã hóa các thông điệp bằng cách thay đổi mỗi chữ cái thành một chữ cái khác hoặc
mỗi khối chữ cái thành một khối chữ cái khác.
Ta sẽ thảo luận về một số cách mã hóa cổ điển, bao gồm cách mã hóa \textit{dịch chuyển}, nghĩa là dịch mỗi chữ cái một số cố định về phía trước
trong bảng chữ cái, rồi quay vòng lại đầu bảng chữ cái nếu cần thiết.
Các cách mã hóa cổ điển mà ta thảo luận là các ví dụ về
mã hóa có khóa bí mật, trong đó, nếu ai đó biết cách mã hóa thì họ cũng có thể
giải mã các thông điệp.
Với mã hóa có khóa, hai bên muốn giao tiếp được phải chia sẻ khóa bí mật.
Các cách mã hóa cổ điển mà ta sẽ thảo luận cũng rất dễ bị phá mã (giải mã được mà không cần khóa).
Chúng tôi sẽ chỉ ra cách giải mã các tin nhắn được gửi bằng cách sử dụng cách mã hóa \textit{dịch chuyển}.

Lý thuyết số cũng rất quan trọng trong mã hóa có khóa công khai, một cách mã hóa
được phát minh vào những năm 1970. Trong mã hóa có khóa công khai, biết cách mã hóa
cũng không chỉ cho ai đó cách giải mã. Hệ thống mã hõa có khóa công khai được sử dụng rộng
rãi nhất, được gọi là hệ mã hóa RSA, mã hóa các thông điệp bằng cách sử
dụng lũy thừa mô-đun, trong đó mô-đun là tích của hai số nguyên tố lớn. Đẻ biết
cách mã hóa đòi hỏi phải có kiến thức về mô-đun và lũy thừa. (Không yêu cầu phải biết
hai hệ số nguyên tố của mô-đun.) Việc biết cách giải mã đòi
hỏi người đó phải biết cách đảo ngược việc mã hóa, điều này chỉ có thể
được thực hiện trong một khoảng thời gian thực tế, khi người đó biết hai thừa số
nguyên tố lớn này. Trong bài viết này, chúng tôi sẽ giải thích cách thức hoạt động
của hệ mã hóa RSA, bao gồm cách mã hóa và giải mã các thông điệp.

Chủ đề của mật mã học cũng bao gồm các giao thức gửi mật mã, là những cách trao
đổi thông điệp được thực hiện bởi hai hoặc nhiều bên để được
bảo mật cụ thể. Ta sẽ thảo luận về hai giao thức quan trọng trong chương
này. Một giao thức cho phép hai người chia sẻ một khóa bí mật chung. Giao thức còn lại có thể được
sử dụng để gửi tin nhắn đã được xác nhận để người nhận có thể chắc chắn rằng chúng đã được
gửi bởi người gửi có chủ đích. Cuối cùng, chúng tôi sẽ giới thiệu về
một hệ mã hóa đồng phôi, hiện đang đóng một vai trò quan trọng trong điện
toán đám mây. Thông tin có thể trở nên dễ bị mất tính bảo mật nếu chúng phải được giải mã
để sử dụng làm đầu vào cho các chương trình. Mã hóa đồng phôi loại bỏ lỗ hổng
này bằng cách cho phép các chương trình được chạy trên dữ liệu được mã hóa. Đầu
ra của các chương trình này sau đó là mã hóa của đầu ra mong muốn.
