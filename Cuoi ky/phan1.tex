\textcolor{blue}{\section{Giới thiệu}}
Lý thuyết số đóng vai trò quan trọng trong mật mã học, chủ thể của việc biến đổi thông tin để thông tin trở nên không dễ dàng khôi phục được nếu không có hiểu biết đặc biệt. Lý thuyết số là cơ sở của nhiều phương pháp mã hóa cổ điển, được sử dụng lần đầu tiên cách đây hàng nghìn năm, và được sử dụng rộng rãi cho đến thế kỷ 20. Các phương pháp mã hóa này mã hóa các thông điệp bằng cách thay đổi mỗi chữ cái thành một chữ cái khác hoặc mỗi khối chữ cái thành một khối chữ cái khác. Ta sẽ thảo luận về một số phương pháp mã hóa cổ điển, bao gồm phương pháp mã hóa \textit{dịch chuyển}, nghĩa là dịch mỗi chữ cái một số cố định về phía trước trong bảng chữ cái, rồi quay vòng lại đầu bảng chữ cái nếu cần thiết. Các phương pháp mã hóa cổ điển mà ta thảo luận là các ví dụ về mã hóa có khóa bí mật, trong đó, nếu ai đó biết cách mã hóa thì họ cũng có thể giải mã. Với mã hóa có khóa bí mật, hai bên muốn giao tiếp được phải chia sẻ khóa bí mật. Các phương pháp mã hóa cổ điển mà ta sẽ thảo luận cũng rất dễ bị phá mã (giải mã được mà không cần khóa). Tôi cũng sẽ chỉ ra cách giải mã các thông điệp được gửi bằng cách sử dụng phương pháp mã hóa \textit{dịch chuyển}.

Lý thuyết số cũng rất quan trọng trong mã hóa có khóa công khai, phương pháp mã hóa được phát minh vào những năm 1970. Trong mã hóa có khóa công khai, biết cách mã hóa cũng không chỉ cho họ cách giải mã. Hệ mã hóa có khóa công khai được sử dụng rộng rãi nhất, được gọi là hệ mã hóa RSA, mã hóa các thông điệp bằng cách sử dụng lũy thừa và modulo của hai số nguyên tố lớn. Để biết cách mã hóa đòi hỏi phải có kiến thức về modulo và lũy thừa. Việc biết cách giải mã đòi hỏi người đó phải biết cách đảo ngược việc mã hóa, điều này chỉ có thể được thực hiện trong một khoảng thời gian ngắn khi người đó biết hai thừa số nguyên tố lớn này. Trong bài viết này, tôi sẽ giải thích cách thức hoạt động của hệ mã hóa RSA, bao gồm cách mã hóa và giải mã các thông điệp.

Chủ đề của mật mã học cũng bao gồm các giao thức gửi mật mã, là những cách trao đổi thông điệp được thực hiện bởi hai hoặc nhiều bên mà vẫn đảm bảo tính bảo mật. Ta sẽ thảo luận về hai giao thức quan trọng trong bài viết này. Một giao thức cho phép hai người chia sẻ chung một khóa bí mật. Giao thức còn lại dùng để gửi tin nhắn đã được xác nhận để người nhận có thể chắc chắn rằng thông điệp được gửi bởi người gửi có chủ đích. Cuối cùng, tôi sẽ giới thiệu về một hệ mã hóa thông tin đồng cấu, hiện đang đóng một vai trò quan trọng trong điện toán đám mây. Thông tin có thể trở nên không được bảo mật nếu chúng phải được giải mã để sử dụng làm đầu vào cho các chương trình. Mã hóa đồng cấu loại bỏ lỗ hổng này bằng cách cho phép các chương trình được chạy trên dữ liệu được mã hóa. Đầu ra của các chương trình này sau đó là mã hóa của đầu ra mong muốn.
