\textcolor{blue}{\section{Mã hóa RSA}}
Để mã hóa thông điệp $M$ sử dụng khóa $(n,e)$, trước tiên, ta thay thể chuỗi $M$ bằng chuỗi số nguyên.
Tương tự như mật mã dịch, ta thay mỗi chữ cái bởi mỗi số tương ứng, nhưng khác biệt ở chỗ số được thay là số được biểu diễn bởi 2 chữ số.
Ví dụ A được thay bằng 00, B được thay bằng 01, ..., J được thay bằng 09.
Tiếp theo, ta chia chuỗi vừa thu được thành các khối có độ dài $2N$, trong đó $N$ là số lớn nhất thỏa mãn số 2525...25 có $2N$ chữ số không vượt quá $n$. Ta có thể thêm vào cuối một vài ký tự (nếu cần thiết) để khối cuối cùng có cùng độ dài với các khổi còn lại.

Sau khi thực hiện các bước trên, ta thu được thông điệp $M$ dưới dạng một chuỗi các số nguyên $m_1, m_2, \ldots, m_k$ ($k>0$).
Quá trình mã hóa sẽ chuyển đổi từng khối $m_i$ thành khối mật mã $c_i$ bằng cách sử dụng hàm
$$ c_i = m_i^e\ \mathbf{mod}\ n$$
Ta để thông điệp mã hóa dưới dạng một dãy các khối số và gửi đến
người nhận mà không thay thế các khổi này bằng cách chữ nữa. Bởi vì hệ RSA mã hóa các khối chữ cái thành các
khối ký tự khác, nên nó là một mật mã khối.

Ví dụ 8 minh họa cách thực hiện mã hóa bằng hệ RSA. Để dễ tính toán, trong ví dụ này, $p$ và $q$ là hai số nguyên tố nhỏ thay vì các số nguyên tố có
300 chữ số trở lên, cho nên nó không có tính bảo mật cao.
Mặc dù mật mã trong ví dụ này không an toàn,
nhưng nó minh họa các kỹ thuật được sử dụng trong hệ mã hóa RSA.
\begin{example}
    Mã hóa thông điệp "STOP" dùng hệ RSA với khóa $(n,13),\ p =43,\ q =59$
\end{example}
\begin{solution}
    Vì $n = pq$ nên ta tính được $n=43\times59 = 2537$, chú ý UCLN$(e,(p-1)(q-1)) = 1$.

    Để mã hóa, trước tiên ta thay các chữ cái trong thông điệp bằng các số tương ứng. Sau đó nhóm thành các nhóm có 4 chữ số (vì $2525 < n < 252525$). Ta được
    \begin{center}
        1819 \hspace{0.5cm} 1415
    \end{center}
    Với mỗi khối, ta dùng hàm
    $$ c = m^{13}\ \mathbf{mod}\ 2537$$
    Tính toán cho thấy $1819^{13}\ \mathbf{mod}\ 2537 = 2081$ và $1415^{13}\ \mathbf{mod}\ 2537 = 2182$.
    Thông điệp mã hóa được gửi đi là 2081 2182.
\end{solution}