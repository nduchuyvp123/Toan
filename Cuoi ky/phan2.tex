\textcolor{blue}{\section{Classical Cryptography}}
Một trong những người sử dụng mật mã được biết đến sớm nhất là Julius Caesar.
Ông đã làm cho các thông điệp trở nên bí mật bằng cách \textit{dịch chuyển} mỗi chữ cái đi ba chữ
cái về phía trước trong bảng chữ cái \thanks{Bảng chữ cái ở đây là bảng chữ cái tiếng Anh, các ví dụ cũng được giữ nguyên gốc} (và ba chữ cái cuối cùng của bảng chữ cái
thành ba chữ cái đầu tiên). Ví dụ, theo sơ đồ này, chữ B được chuyển thành E và
chữ X được chuyển thành A. Đây là một ví dụ về mã hóa, tức là quá trình tạo một thông
điệp bí mật.

Để biểu diễn quy trình mã hóa của Caesar theo toán học, trước tiên
hãy thay thế mỗi chữ cái bằng một phần tử của $ \mathbf{Z}_{26} $, nghĩa là một số nguyên từ 0
đến 25 dựa trên vị trí của nó trong bảng chữ cái. Ví dụ, thay A bằng 0,
K bằng 10 và Z bằng 25. Phương pháp mã hóa của Caesar có thể được biểu diễn bằng
hàm $f$ tác động vào số nguyên không âm $p$, $0 \leq p \leq  25$, số nguyên $f(p)$ trong tập
$\{0, 1, 2, \ldots, 25\}$ sao cho:
$$f(p) = (p+3)\ \mathbf{mod}\ 26$$
Như vậy, trong phiên bản đã được mã hóa của thông điệp, chữ cái được biểu diễn bởi $p$ sẽ được thay thế bằng
chữ cái được biểu diễn bởi $(p+3)\ \mathbf{mod}\ 26$
\begin{example}
    Thông điệp mã hóa của thông điệp "MEET YOU IN THE PARK" sử dụng cách mã hóa Caesar là gì?
\end{example}
\begin{solution}
    Trước tiên, ta thay các chữ cái trong thông điệp bởi các số tương ứng, ta được:
    \begin{center}
        12 4 4 19 \hspace{0.5cm} 24 14 20 \hspace{0.5cm} 8 13 \hspace{0.5cm} 19 7 4 \hspace{0.5cm} 15 0 17 10
    \end{center}
    Tiếp theo, ta thay mỗi số $p$ bởi $f(p) = (p+3)\ \mathbf{mod}\ 26$, ta được:
    \begin{center}
        15 7 7 22 \hspace{0.5cm} 1 17 23 \hspace{0.5cm} 11 16 \hspace{0.5cm} 22 10 7 \hspace{0.5cm} 18 3 20 13
    \end{center}
    Với mỗi số, ta thay ngược lại bởi các chữ cái tương ứng, ta thu được thông điệp được mã hóa "PHHW BRX LQ WKH SDUN".
\end{solution}
Để phục hồi lại thông điệp gốc đã được mã hóa theo cách mã hóa Caesar, ta cần phải dùng hàm ngược $f^{-1}$ của hàm $f$.
Chú ý hàm $f^{-1}$ ánh xạ số nguyên $p$ từ tập $ \mathbf{Z}_{26} $, đến $f^{-1}(p) = (p-3)\ \mathbf{mod}\ 26$.
Nói cách khác, để tìm được thông điệp gốc, mỗi chữ cái được \textit{dịch chuyển} ba chữ
cái về phía sau trong bảng chữ cái, với ba chữ cái đầu tiên được chuyển thành ba chữ cái cuối
cùng. Quá trình xác định thông điệp gốc từ thông điệp được mã
hóa được gọi là \textbf{giải mã}.

Có nhiều cách khác nhau để khái quát cách mã hóa Caesar. Ví dụ, thay vì \textit{dịch chuyển} mỗi chữ cái cho 3,
ta có thể \textit{dịch chuyển} mỗi chữ cái cho $k$ chữ cái về phía trước, tức là:
$$f(p) = (p+k)\ \mathbf{mod}\ 26$$
Một mật mã như thế được gọi là \textbf{mật mã dịch}. Đối với loại mật mã này, việc giải mã thực hiện bằng cách dùng
$$f^{-1}(p) = (p-k)\ \mathbf{mod}\ 26$$
Ở đây số nguyên $k$ được gọi là \textbf{khóa}. Chúng tôi minh họa việc sử dụng mật mã dịch trong Ví dụ 2 và 3.
\begin{example}
    Mã hóa văn bản thông báo “STOP GLOBAL WARMING” bằng cách sử dụng mật mã dịch với khóa $k = 11$
\end{example}
\begin{solution}
    Để mã hóa thông điệp “STOP GLOBAL WARMING”, trước tiên ta thay các chữ cái trong thông điệp bởi các số tương ứng trong $ \mathbf{Z}_{26} $. Điều này tạo ra chuỗi
    \begin{center}
        18 19 14 15 \hspace{0.5cm} 6 11 14 1 0 11 \hspace{0.5cm} 22 0 17 12 8 13 6
    \end{center}
    Bây giờ ta áp dụng hàm dịch $f(p) = (p+k)\ \mathbf{mod}\ 26$ cho mỗi số trong chuỗi này. Ta thu được
    \begin{center}
        3 4 25 0 \hspace{0.5cm} 17 22 25 12 11 22 \hspace{0.5cm} 7 11 2 23 19 24 17
    \end{center}
    Thay thế chuỗi cuối cùng này trở lại các chữ cái, ta thu được thông điệp mã hóa
    "DEZA RWZMLW HLCXTYR".
\end{solution}
\begin{example}
    Giải mã thông điệp mật mã "LEWLYPLUJL PZ H NYLHA ALHJOLY" đã được mã
    hóa bằng mật mã dịch với khóa $k = 7$.
\end{example}
\begin{solution}
    Để giải mã thông điệp "LEWLYPLUJL PZ H NYLHA ALHJOLY", trước tiên ta
    thay các chữ cái bởi các phần tử tương ứng trong $ \mathbf{Z}_{26} $. Ta đạt được
    \begin{center}
        11 4 22 11 24 15 11 20 9 11 \hspace{0.5cm} 15 25 \hspace{0.5cm} 7 \hspace{0.5cm} 13 24 11 7 0 \hspace{0.5cm} 0 11 7 9 14 11 24
    \end{center}
    Để giải mã, ta phải dùng hàm $f^{-1}(p) = (p-7)\ \mathbf{mod}\ 26$, áp dụng cho mỗi số trong chuỗi, ta được
    \begin{center}
        4 23 15 4 17 8 4 13 2 4 \hspace{0.5cm} 8 18 \hspace{0.5cm} 0 \hspace{0.5cm} 6 17 4 0 19 \hspace{0.5cm}  19 4 0 2 7 4 17
    \end{center}
    Cuối cùng, ta thay thế các số này trở lại thành các chữ cái để thu được thông điệp gốc. Ta thu được "EXPERIENCE IS A GREAT TEACHER".
\end{solution}

Rõ ràng, phương pháp của Caesar và mật mã dịch không có độ an toàn cao.
Ta có thể tổng quát hóa mật mã dịch để tăng cường độ bảo mật
bằng cách sử dụng hàm
$$f(p) = (ap+b)\ \mathbf{mod}\ 26$$
trong đó $a$ và $b$ là các số nguyên sao cho thỏa mãn $f$ là song ánh ($f(p)$ là song ánh khi và chỉ khi
UCLN$(a,26)$ = 1). Một hàm như vậy được gọi là \textbf{hàm tuyến tính} và mật mã dịch thu được sau khi dùng hàm tuyến tính là \textbf{mật mã dich tuyến tính}.
\begin{example}
    Chữ cái nào thay thế cho chữ K khi dùng hàm $f(p) = (7p + 3)\ \textbf{mod}\ 26$ để mã hóa?
\end{example}
\begin{solution}
    Đầu tiên, ta biết rằng số 10 tương ứng cho chữ K. Sau đó, dùng hàm mã hóa
    $f(p)$, ta được $f (10) = (7 \times 10 + 3)\ \textbf{mod}\ 26 = 21$. Bởi vì 21 tương ứng
    với chữ V, vậy nên sẽ K được thay thế bằng V trong thông điệp mã hóa.
\end{solution}