\textcolor{blue}{\section{Giải mã RSA}}
Để giải mã một thông điệp được mã hóa bằng hệ RSA, vì là hệ mã hóa khóa công khai nên ta phải có khóa giải mã $(n,d)$ mới có thể giải mã được.
Chú ý rằng UCLN$(e,\phi(n)) = 1$ nên luôn tồn tại $d$ để $de \equiv 1 (\mathbf{mod}\ \phi(n))$.
Gọi $de = 1 + k(p-1)(q-1)$ với $k$ là số nguyên. Suy ra
$$c^d \equiv (m^e)^d = m^{de} = m^{1 + k(p-1)(q-1)}\ (\mathbf{mod}\ n)$$
Giả sử $m$ không chia hết cho $p$ và $q$, từ định lý Fermat nhỏ \footnote{Định lý Fermat nhỏ: Nếu $p$ là số nguyên tố và $a$ là số tự nhiên không chia hết cho $p$ thì $ a^{p-1} \equiv 1 (\mathbf{mod}\ p)$},
ta có $ m^{p-1} \equiv 1 (\mathbf{mod}\ p)$ và $ m^{q-1} \equiv 1 (\mathbf{mod}\ q)$. Dẫn đến
$$ c^d \equiv m \cdot (m^{p-1})^{k(q-1)} \equiv m \cdot 1 = m\ (\mathbf{mod}\ p)$$
và
$$ c^d \equiv m \cdot (m^{q-1})^{k(p-1)} \equiv m \cdot 1 = m\ (\mathbf{mod}\ q)$$
Vì UCLN$(p,q)=1$ nên
$$ c^d \equiv m\ (\mathbf{mod}\ pq)$$
Ví dụ 9 minh họa cách giải mã thông điệp được mã hóa bằng hệ RSA.

\begin{example}
    Giải mã thông điệp 0981 0461. Biết rằng thông điệp được mã hóa bằng mật mã RSA và khóa mã hóa là $(n,13),\ p=43,\ q=59$ (giống Ví dụ 8)
\end{example}
\begin{solution}
    Ta có $n = 43 \times 59 = 2537$ và $e = 13$

    Khóa để giải mã là cặp số $(n,d)$ trong đó $n=pq$ và $d = \overline{e}$ hay  $de \equiv 1 (\mathbf{mod}\ n)$

    Với $n=2537,\ e = 13$ ta tìm được $d=937$. Ta dùng (2537,937) làm khóa giải mã.
    Với mỗi số $c$ trong mật mã, ta giải mã để được $m$ như sau
    $$ m = c^d\ \mathbf{mod}\ n$$
    Ở ví dụ này
    $$ m = c^{937}\ \mathbf{mod}\ 2537$$
    Tuy $c^d$ là một số lớn, nhưng với sự trợ giúp của một số thuật toán và máy tính, ta vẫn tính được.
    Tiếp theo, ta có kết quả tính toán $0981^{937}\ \mathbf{mod}\ 2537 = 0704$ và $0461^{937}\ \mathbf{mod}\ 2537 = 1115$.
    Vậy thông điệp gốc là 0704 1115. Thay thế bằng các chữ cái tiếng Anh tương ứng, ta được nội dung thông điệp là "HELP".
\end{solution}