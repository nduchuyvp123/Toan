\section{Mở đầu}

Về tính phẳng của đồ thị, liệu đồ thị có thể được vẽ trên một mặt phẳng theo cách không có cạnh nào cắt nhau hay không,
là một tính chất thú vị cần khảo sát. Với một vài định lý đơn giản, có thể thấy rằng $K_5$ và $K_{3,3}$ là đồ thị không phẳng.
Kuratowski đã sử dụng quan sát gần như dễ dàng này thành một định lý mạnh mẽ cho thấy điều kiện cần và đủ của tính phẳng.
Để chứng minh định lý này, ta cần sử dụng một vài các định lý, hệ quả và bổ đề.
Trong bài viết này, chúng tôi bắt đầu với lý thuyết đồ thị cơ bản, sau đó tới các khái niệm và định lý liên quan đến đồ thị phẳng.
Trong phần cuối cùng, chúng tôi sẽ đưa ra một cách chứng minh định lý Kuratowski.