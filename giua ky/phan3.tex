\section{Định lý Kuratowski}
Năm 1930, Kuratowski công bố định lý đưa ra một điều kiện cần và đủ cho tính phẳng.
\begin{theorem}[Kuratowski]
    \label{thr:kuratowski}
    Một đồ thị phẳng khi và chỉ khi không chứa bất kỳ đồ thị con nào là đồ thị phân chia của $K_5$ hoặc $K_{3,3}$
\end{theorem}
\begin{figure}[H]
    \centering
    \begin{minipage}{0.05\textwidth}

    \end{minipage}
    \hfill
    \begin{minipage}{0.55\textwidth}

        \begin{tikzpicture}
            \draw[black, thick] (0,1) -- (1,0);
            \draw[black, thick] (0,1) -- (2,2);
            \draw[black, thick] (3,4) -- (2,2);
            \draw[black, thick] (3,4) -- (1,4);
            \draw[black, thick] (0,3) -- (1,4);
            \draw[black, thick] (4,3) -- (3,4);
            \draw[black, thick] (1,0) -- (4,3);
            \draw[black, thick] (1,0) -- (3,0);
            \draw[black, thick] (3,0) -- (4,1);
            \draw[black, thick] (3,0) -- (0,3);
            \draw[black, thick] (2,2) -- (4,1);
            \draw[black, thick] (0,1) -- (0,3);
            \draw[black, thick] (4,1) -- (4,3);
            \draw[black, thick] (0,3) -- (4,3);

            \filldraw[black] (1,0) circle (2pt);
            \filldraw[black] (0,1) circle (2pt);
            \filldraw[black] (3,0) circle (2pt);
            \filldraw[black] (4,1) circle (2pt);
            \filldraw[black] (2,2) circle (2pt);
            \filldraw[black] (0,3) circle (2pt);
            \filldraw[black] (4,3) circle (2pt);
            \filldraw[black] (1,4) circle (2pt);
            \filldraw[black] (3,4) circle (2pt);

            \node at (2, -1, 0) {Đồ thị không phẳng $G$};
        \end{tikzpicture}

    \end{minipage}
    \hfill
    \begin{minipage}{0.325\textwidth}
        \label{fig:G1}
        \begin{tikzpicture}
            \draw[black, thick] (0,1) -- (1,0);
            \draw[black, thick] (0,1) -- (2,2);
            \draw[black, thick] (3,4) -- (2,2);
            \draw[black, thick] (3,4) -- (1,4);
            \draw[black, thick] (0,3) -- (1,4);
            \draw[black, thick] (4,3) -- (3,4);
            \draw[black, thick] (1,0) -- (4,3);
            \draw[black, thick] (1,0) -- (3,0);
            \draw[black, thick] (3,0) -- (4,1);
            \draw[black, thick] (3,0) -- (0,3);
            \draw[black, thick] (2,2) -- (4,1);
            \draw[black, thick] (0,1) -- (0,3);
            \draw[black, thick] (4,1) -- (4,3);
            \draw[black, thick] (0,3) -- (4,3);

            \filldraw[blue] (1,0) circle (3pt);
            \filldraw[purple] (0,1) circle (3pt);
            \filldraw[purple] (3,0) circle (3pt);
            \filldraw[blue] (4,1) circle (3pt);
            \filldraw[black] (2,2) circle (2pt);
            \filldraw[blue] (0,3) circle (3pt);
            \filldraw[purple] (4,3) circle (3pt);
            \filldraw[black] (1,4) circle (2pt);
            \filldraw[black] (3,4) circle (2pt);

            \node at (2,-1,0) {Đồ thị không phẳng $G$};
        \end{tikzpicture}

    \end{minipage}
\end{figure}

\begin{figure}[H]
    \centering

    \begin{tikzpicture}
        \draw[black, thick] (0,1) -- (1,0);
        \draw[black, thick] (1,0) -- (4,3);
        \draw[black, thick] (1,0) -- (3,0);
        \draw[black, thick] (3,0) -- (4,1);
        \draw[black, thick] (3,0) -- (0,3);
        \draw[black, thick] (0,1) -- (0,3);
        \draw[black, thick] (4,1) -- (4,3);
        \draw[black, thick] (0,3) -- (4,3);
        \draw[black, thick] (0,1) -- (2,2);
        \draw[black, thick] (2,2) -- (4,1);
        \filldraw[blue] (1,0) circle (3pt);
        \filldraw[purple] (0,1) circle (3pt);
        \filldraw[purple] (3,0) circle (3pt);
        \filldraw[blue] (4,1) circle (3pt);
        \filldraw[blue] (0,3) circle (3pt);
        \filldraw[purple] (4,3) circle (3pt);
        \filldraw[black] (2,2) circle (2pt);
        \draw [-{To}] (6.5, 2) -- (7.5, 2);
        \node at (2,-2,0) {Đồ thị con của $G$};
    \end{tikzpicture}
    \hspace{2cm}
    \begin{tikzpicture}
        \draw[black, thick] (8,-0.5) -- (12,-0.5);
        \draw[black, thick] (8,-0.5) -- (12,1.5);
        \draw[black, thick] (8,-0.5) -- (12,3.5);
        \draw[black, thick] (8,1.5) -- (12,-0.5);
        \draw[black, thick] (8,1.5) -- (12,1.5);
        \draw[black, thick] (8,1.5) -- (12,3.5);
        \draw[black, thick] (8,3.5) -- (12,-0.5);
        \draw[black, thick] (8,3.5) -- (12,1.5);
        \draw[black, thick] (8,3.5) -- (12,3.5);
        \filldraw[white] (10,5) circle (2pt);
        \filldraw[black] (10,3.5) circle (2pt);
        \filldraw[blue] (12,-0.5) circle (3pt);
        \filldraw[purple] (8,-0.5) circle (3pt);
        \filldraw[blue] (12,1.5) circle (3pt);
        \filldraw[purple] (8,1.5) circle (3pt);
        \filldraw[blue] (12,3.5) circle (3pt);
        \filldraw[purple] (8,3.5) circle (3pt);
        \node at (10,-2,0) {Đồ thị phân chia của $K_{3,3}$};
    \end{tikzpicture}
\end{figure}

Để chứng minh định lý này, ta cần chứng minh một số bổ đề cần thiết.
